\documentclass{article}
\usepackage[utf8]{inputenc}
\usepackage{amsmath, amssymb, amsthm}

\newtheorem{theorem}{Theorem}

\newtheorem*{definition}{Definition}
\newtheorem*{corollary}{Corollary}

\begin{document}

\section{Introduction}

The algebraic descriptions of vector addition and scalar multiplication for vectors in a plane yield the following properties:
\begin{enumerate}
	\item For all vectors \textit{x} and \textit{y}, $x + y = y + x$.
	\item For all vectors \textit{x}, \textit{y}, and \textit{z}, $(x + y) + z = x + (y + z)$.
	\item There exists a vector denoted \textit{0} such that $x + \textit{0} = x$ for each vector \textit{x}.
	\item For each vector \textit{x} there is a vector \textit{y} such that $x + y = \textit{0}$.
	\item For each vector \textit{x}, $1x = x$.
	\item For each pair of real numbers \textit{a} and \textit{b} and each vector \textit{x}, $(ab)x = a(bx)$.
	\item For each real number \textit{a} and each pair of vectors \textit{x} and \textit{y}, $a(x + y) = ax + ay$.
	\item For each pair of real numbers \textit{a} and \textit{b} and each vector \textit{x}, $(a + b)x = ax + bx$.	
\end{enumerate}

Any mathematical structure possessing these eight properties is called a \textit{vector space}.

\section{Vector Spaces}

\begin{definition}
	A \textbf{vector space} (or \textbf{linear space}) \textbf{\textup{V}} over a field F consists of a set on which two operations (called \textbf{addition} and \textbf{scalar multiplication}, respectively) are defined so that for each pair of elements x, y, in \textbf{\textup{V}} there is a unique element $x + y$ in \textbf{\textup{V}}, and for each element a in F and each element x in \textbf{\textup{V}} there is a unique element ax in \textbf{\textup{V}}, such that the following conditions hold:
	\begin{enumerate}
		\setlength{\itemindent}{.3in}
		\item[(VS 1)] For all x, y in \textbf{\textup{V}}, $x + y = y + x$ (commutativity of addition).
		\item[(VS 2)] For all x, y, z in \textbf{\textup{V}}, $(x + y) + z = x + (y + z)$ (associativity of addition).
		\item[(VS 3)] There exists an element in \textbf{\textup{V}} denoted by 0 such that $x + \textit{0} = x$ for each x in \textbf{\textup{V}}.
		\item[(VS 4)] For each element x in \textbf{\textup{V}} there exists and element y in \textbf{\textup{V}} such that $x + y = \textit{0}$.
		\item[(VS 5)] For each element x in \textbf{\textup{V}}, $1x = x$.
		\item[(VS 6)] For each pair of elements a, b in F and each element x in \textbf{\textup{V}}, $(ab)x = a(bx)$.
		\item[(VS 7)] For each element a in F and each pair of elements x, y in \textbf{\textup{V}}, $a(x + y) = ax + ay$.
		\item[(VS 8)] For each pair of elements a, b in F and each element x in \textbf{\textup{V}}, $(a + b)x = ax + bx$.
	\end{enumerate}
	The elements x + y and ax are called the \textbf{sum} of x and y and the \textbf{product} of a and x, respectively.
\end{definition}

The elements of the field \textit{F} are called \textbf{scalars} and the elements of the vector space \textbf{V} are called \textbf{vectors}.

An object of the form $(a_1, a_2, ..., a_n)$ where the entries $a_1, a_2, ..., a_n$ are elements of a field \textit{F}, is called an \textbf{$n$-tuple} with entries from \textit{F}. The elements $a_1, a_2, ..., a_n$ are called the \textbf{entries} or \textbf{components} of the $n$-tuple. Two $n$-tuples $(a_1, a_2, ..., a_n)$ and $(b_1, b_2, ..., b_n)$ are called \textbf{equal} if $a_i = b_i$ for $i = 1, 2, ..., n$.

An $m \times n$ \textbf{matrix} with entries from a field \textit{F} is a rectangular array of the form
\[\begin{bmatrix}
	a_{11} & a_{12} & \hdots & a_{1n} \\
	a_{21} & a_{22} & \hdots & a_{2n} \\
	\vdots & \vdots & & \vdots \\
	a_{m1} & a_{m2} & \hdots & a_{mn}
\end{bmatrix}\]
where each entry $a_{ij}$ is an element of \textit{F}. We call the entries $a_{ij}$ with $i = j$ the \textbf{diagonal entries} of the matrix.

The $m \times n$ matrix in which each entry equals zero is called the \textbf{zero matrix} and is denoted by \textit{O}.

If the number of rows and columns of a matrix are equal, the matrix is called \textbf{square}.

A \textbf{polynomial} with coefficients from a field \textit{F} is an expression of the form \[f(x) = a_nx^n + a_{n-1}x^{n-1} + \hdots + a_1x + a_0,\] where $n$ is a nonnegative integer and each $a_k$, called the \textbf{coefficient} of $x^k$, is in \textit{F}. If $f(x) = 0$, then $f(x)$ is called the \textbf{zero polynomial} and, for convenience, its degree is defined to be $-1$; otherwise, the \textbf{degree} of a polynomial is defined to be the largest exponent of $x$ that appears in the representation with a nonzero coefficient.

\begin{theorem}[\textbf{Cancellation Law for Vector Addition}]
	If $x$, $y$, and $z$ are vectors in a vector space \textbf{\textup{V}} such that $x + z = y + z$ then $x = y$.
\end{theorem}

The vector \textit{0} in (VS 3) is called the \textbf{zero vector} of \textbf{V}, and the vector $y$ in (VS 4) is called the \textbf{additive inverse} of $x$ and is denoted by $-x$.

\begin{theorem}
	In any vector space \textbf{\textup{V}}, the following statements are true:
	\begin{enumerate}
		\item[(a)] $0x = 0$ for each $x\in\textbf{\textup{V}}$.
		\item[(b)] $(-a)x = -(ax) = a(-x)$ for each $a\in F$ and each $x\in\textbf{\textup{V}}$.
		\item[(c)] $a\textit{0} = \textit{0}$ for each $a\in F$.
	\end{enumerate}
\end{theorem}

\pagebreak
\subsection*{Exercises}
\subsubsection*{1. Label the following statements as true or false.}
\begin{enumerate}
	\item[(a)] Every vector space contains a zero vector - \textbf{True}.
	
	It is included in the definition of a vector space (VS 3).
	\item[(b)] A vector space may have more than one zero vector - \textbf{False}.
	
	Suppose there were two such vectors, $x$ and $y$, and one nonzero vector $z$. Then $x + z = z = y + z$, and $x + (z + (-z)) = x = y + (z + (-z)) = y$. 
	\item[(c)] In any vector space, $ax = bx$ implies that $a = b$ - \textbf{False}.
	
	Consider $x = \textit{0}$ but $a \ne b$.
	\item[(d)] In any vector space, $ax = ay$ implies that $x = y$ - \textbf{False}.
	
	Consider $a = 0$ but $x \ne y$.
	\item[(e)] A vector in $F^n$ may be regarded as a matrix in $M_{n\times 1}(F)$ - \textbf{True}.
	\item[(f)] An $m \times n$ matrix has $m$ columns and $n$ rows - \textbf{False}.
	
	An $m \times n$ matrix has $m$ rows and $n$ columns.
	\item[(g)] In $P(F)$, only polynomials of the same degree may be added - \textbf{False}.
	
	Not true based on the definition of addition in $P(F)$.
	\item[(h)] If $f$ and $g$ are polynomials of degree $n$, then $f + g$ is a polynomial of degree $n$ - \textbf{False}.
	
	Consider $x$ and $-x$.
	\item[(i)] If $f$ is a polynomial of degree $n$ and $c$ is a nonzero scalar, then $cf$ is a polynomial of degree $n$ - \textbf{True}.
	
	Follows from definition of scalar multiplication in $P(F)$.
	\item[(j)] A nonzero scalar of \textit{F} may be considered to be a polynomial in $P(F)$ having degree zero - \textbf{True}.
	
	If $a$ is a nonzero scalar, it can be expressed as $ax^0$.
	\item[(k)] Two functions in $\mathcal{F}(S, F)$ are equal if and only if the have the same value at each element of $S$ - \textbf{True}.
	
	By definition, two functions $f$, $g$ in $\mathcal{F}(S, F)$ are equal when $f(x) = g(x)$ for each $x$ in F.
\end{enumerate}

\subsubsection*{2. Write the zero vector of $M_{3 \times 4}(F)$.}
\[\begin{bmatrix}
	0 & 0 & 0 & 0 \\
	0 & 0 & 0 & 0 \\
	0 & 0 & 0 & 0
\end{bmatrix}\]

\subsubsection*{8. In any vector space \textbf{V}, show that $(a + b)(x + y) = ax + ay + bx + by$ for any $x, y \in \textbf{\textup{V}}$ and any $a, b \in F$.}
$(a + b)(x + y) = (a + b)x + (a + b)y = ax + bx + ay + by$.

\subsubsection*{9. Prove Corollaries 1 and 2 of Theorem 1.1 and Theorem 1.2(c).}
\begin{corollary}
	The vector 0 described in (VS 3) is unique.
\end{corollary}
\begin{proof}
	Suppose that there are vectors $x, y, z \in \textbf{\textup{V}}$ such that $x + z = y + z = z$. Then $x = x + 0 = x + (z + (-z)) = (x + z) + (-z) = (y + z) + (-z) = y + (z + (-z)) = y + 0 = y$.
\end{proof}
\begin{corollary}
	The vector y described in (VS 4) is unique.
\end{corollary}
\begin{proof}
	Suppose that there are vectors $x, y, z \in \textbf{\textup{V}}$ such that $x + y = x + z = 0$. Then $y = 0 + y = x + (-x) + y = (x + y) + (-x) = (x + z) + (-x) = x + (-x) + z = 0 + z = z$.
\end{proof}

\subsubsection*{11. Let $\textbf{\textup{V}} = \{\textit{0}\}$ consist of a single vector \textit{0} and define $\textit{0} + \textit{0} = \textit{0}$ nad $c\textit{0} = \textit{0}$ for each scalar $c$ in $F$. Prove that \textbf{V} is a vector space over $F$. (\textbf{V} is called the \textbf{zero vector space}.)}
\begin{proof}
	For any $x,y,z \in \textbf{\textup{V}}$ and $a,b \in F$:
	\begin{enumerate}
		\item $x + y = 0 + 0 = y + x$ (VS 1)
		\item $(x + y) + z = (0 + 0) + 0 = 0 + (0 + 0) = x + (y + z)$ (VS 2)
		\item $x + 0 = 0 + 0 = 0 = x$ (VS 3)
		\item $x + y = 0 + 0 = 0$ (VS 4)
		\item $1x = 1 \times 0 = 0 = x$ (VS 5)
		\item $(ab)x = (ab) \times 0 = 0 = a (b \times 0) = a(bx)$ (VS 6)
		\item $a(x + y) = a(0 + 0) = 0 + 0 = a \times 0 + a \times 0 = ax + ay$ (VS 7)
		\item $(a + b)x = (a + b) \times 0 = 0 = 0 + 0 = a \times 0 + b \times 0 = ax + bx$ (VS 8)
	\end{enumerate}
	Therefore \textbf{V} satisfies all conditions necessary for it to be a vector space.
\end{proof}

\pagebreak

\subsubsection*{13. Let \textbf{V} denote the set of ordered pairs of real numbers. If $(a_1, a_2)$ and $(b_1, b_2)$ are elements of \textbf{V} and $c \in R$, define \[(a_1, a_2) + (b_1, b_2) = (a_1 + b_1, a_2b_2)\] and \[c(a_1, a_2) = (ca_1, a_2).\] Is \textbf{V} a a vector space over $R$ with these operations?}
\begin{proof}
	Let $(x_1, x_2) \in \textbf{\textup{V}}$ and $a,b \in R$. Then \[(a + b)(x_1, x_2) = ((a + b)x_1, x_2) = (ax_1 + bx_1, x_2)\]and \[a(x_1, x_2) + b(x_1, x_2) = (ax_1, x_2) + (bx_1, x_2) = (ax_1 + bx_1, x_2^2)\]so \[(a + b)(x_1, x_2) \ne a(x_1, x_2) + b(x_1, x_2)\]so \textbf{V} is not a vector space over $R$.
\end{proof}

\subsubsection*{14. Let $\textbf{\textup{V}} = \{(a_1, a_2, \dots, a_n): a_i \in C \text{ for } i=1,2,\dots,n$\}; so \textbf{V} is a vector space over $C$ by Example 1. Is \textbf{V} a vector space over the field of real numbers with the operations of coordinatewise addition and multiplication?}
\begin{proof}
	Notice that any number $x \in R$ can be expressed as $x + 0i$ in $C$, so if \textbf{V} is a vector space over $C$, it is also a vector space over $R$.
\end{proof}

\subsubsection*{15. Let $\textbf{\textup{V}} = \{(a_1, a_2, \dots, a_n): a_i \in R \text{ for } i=1,2,\dots,n\}$; so \textbf{V} is a vector space over $R$ by Example 1. Is \textbf{V} a vector space over the field of complex numbers with the operations of coordinatewise addition and scalar multiplication?}
\begin{proof}
	Consider $c = x + yi$ and $a = (a_1)$ with $y,a_1 \ne 0$. Then $ca = (x + yi)(a_1) = (xa_1 + ya_1i)$, so the entries of ca aren't in $R$, so \textbf{V} is not a vector space over $C$.
\end{proof}

\end{document}